\documentclass[11pt, a4paper]{article}
%\usepackage{geometry}
\usepackage[inner=1.5cm,outer=1.5cm,top=2.5cm,bottom=2.5cm]{geometry}
\pagestyle{empty}
\usepackage{longtable}
\usepackage{array,booktabs,enumitem} 
\newcolumntype{P}[1]{>{\endgraf\vspace*{-\baselineskip}}p{#1}}
\usepackage{graphicx}
\usepackage{fancyhdr, lastpage, bbding, pmboxdraw}
\usepackage[usenames,dvipsnames]{color}
\definecolor{darkblue}{rgb}{0,0,.6}
\definecolor{darkred}{rgb}{.7,0,0}
\definecolor{darkgreen}{rgb}{0,.6,0}
\definecolor{red}{rgb}{.98,0,0}
\usepackage[colorlinks,pagebackref,pdfusetitle,urlcolor=darkblue,citecolor=darkblue,linkcolor=darkred,bookmarksnumbered,plainpages=false]{hyperref}
\renewcommand{\thefootnote}{\fnsymbol{footnote}}

\pagestyle{fancyplain}
\fancyhf{}
\lhead{ \fancyplain{}{Teknologi Open Source} }
%\chead{ \fancyplain{}{} }
\rhead{ \fancyplain{}{\today} }
%\rfoot{\fancyplain{}{page \thepage\ of \pageref{LastPage}}}
\fancyfoot[RO, LE] {page \thepage\ of \pageref{LastPage} }
\thispagestyle{plain}

%%%%%%%%%%%% LISTING %%%
\usepackage{listings}
\usepackage{caption}
\DeclareCaptionFont{white}{\color{white}}
\DeclareCaptionFormat{listing}{\colorbox{gray}{\parbox{\textwidth}{#1#2#3}}}
\captionsetup[lstlisting]{format=listing,labelfont=white,textfont=white}
\usepackage{verbatim} % used to display code
\usepackage{fancyvrb}
\usepackage{acronym}
\usepackage{amsthm}
\VerbatimFootnotes % Required, otherwise verbatim does not work in footnotes!


\definecolor{OliveGreen}{cmyk}{0.64,0,0.95,0.40}
\definecolor{CadetBlue}{cmyk}{0.62,0.57,0.23,0}
\definecolor{lightlightgray}{gray}{0.93}



\lstset{
%language=bash,                          % Code langugage
basicstyle=\ttfamily,                   % Code font, Examples: \footnotesize, \ttfamily
keywordstyle=\color{OliveGreen},        % Keywords font ('*' = uppercase)
commentstyle=\color{gray},              % Comments font
numbers=left,                           % Line nums position
numberstyle=\tiny,                      % Line-numbers fonts
stepnumber=1,                           % Step between two line-numbers
numbersep=5pt,                          % How far are line-numbers from code
backgroundcolor=\color{lightlightgray}, % Choose background color
frame=none,                             % A frame around the code
tabsize=2,                              % Default tab size
captionpos=t,                           % Caption-position = bottom
breaklines=true,                        % Automatic line breaking?
breakatwhitespace=false,                % Automatic breaks only at whitespace?
showspaces=false,                       % Dont make spaces visible
showtabs=false,                         % Dont make tabls visible
columns=flexible,                       % Column format
morekeywords={__global__, __device__},  % CUDA specific keywords
}

\newenvironment{myitemize}
{ \begin{itemize} [label={--},noitemsep,leftmargin=*,topsep=0pt,partopsep=0pt]  }
	{ \end{itemize}  }
	      
%%%%%%%%%%%%%%%%%%%%%%%%%%%%%%%%%%%%
\begin{document}
\begin{center}
{\Large \textsc{Teknologi Open Source}}
\end{center}
\begin{center}
Semester Gasal 2016
\end{center}
%\date{10 Agustus 2014}

\begin{center}
\rule{6in}{0.4pt}
\begin{minipage}[t]{.75\textwidth}
\begin{tabular}{llcccll}
\textbf{Dosen:} & Iwan Njoto Sandjaja & & &  & \textbf{Waktu:} & Kamis 07:30 -- 10:00 \\
\textbf{Email:} &  \href{mailto:iwanns@petra.ac.id}{iwanns@petra.ac.id} & & & & \textbf{Tempat:} & Lab. Sistem Informasi.
\end{tabular}
\end{minipage}
\rule{6in}{0.4pt}
\end{center}
\vspace{.5cm}
\setlength{\unitlength}{1in}
\renewcommand{\arraystretch}{2}

\noindent\textbf{Laman Kuliah:} \begin{enumerate}
\item \url{https://github.com/inyoot/TOS/}
\end{enumerate}

\vskip.15in
\noindent\textbf{Jam Kantor:} Setelah kelas, atau dengan membuat janji temu, atau bertanya lewat LINE grup/email.

\vskip.15in
\noindent\textbf{Referensi Utama:} %\footnotemark
Berikut ini adalah daftar buku yang menarik dan berguna untuk pembahasan selama kuliah:
\begin{itemize}
\item Wesley J. Chun, {\textit{Core Python Application Programming 3rd edition}}, Prentice Hall, 2012.
\end{itemize}

% \footnotetext{Downloadable ebook versions are available on AeLP.}

\vskip.15in
\noindent\textbf{Tujuan:}  
\begin{itemize}
	\item Mengenalkan mahasiswa ke teknologi open source yang banyak digunakan di kalangan akademik hingga aplikasi enterprise.
	\item Mengenalkan mahasiswa ke sistem operasi open source Linux Ubuntu, dan aplikasi-aplikasi yang ada sebagai alternatif sistem proprietary.
	\item Mengenalkan bahasa pemrograman Python sebagai salah satu bahasa populer di dunia open source.
\end{itemize}

\vskip.15in
\noindent\textbf{Kompetensi yang akan dicapai:}  
\begin{itemize}
	\item Kemampuan untuk mengadministrasi sistem operasi berbasis open source.
	\item Kemampuan untuk mengkompilasi dan mengeksekusi program berbasis open source.
	\item Kemampuan mendeploy aplikasi enterprise berbasis open source.
	\item Kemampuan mendeploy aplikasi berbasis web.
\end{itemize}

\vskip.15in
\noindent\textbf{Susunan Materi:}

Bahan Kajian: System Administration and Maintenance (SAM), Web
Technologies and Development (WTD), Enterprise Development Software
(EDS).

\begin{enumerate}
	\def\labelenumi{\arabic{enumi}.}
	\item
	Dasar-dasar Open Source: mengenalkan teknologi open source dan
	lisensinya.
	\item
	SAM/Operating Systems: Installation, Configuration. Instalasi Linux
	Ubuntu: mengenalkan berbagai jenis instalasi dan langkah-langkahnya.
	\item
	SAM/Applications: Installation, Configuration. Administrative
	Activities: User and group management. Perintah-perintah dasar Linux:
	mengenalkan perintah-perintah command line yang umum digunakan dalam
	mengadministrasi sistem operasi Linux dengan shell scripting.
	\item
	SAM/Administrative Activities: Automation management. Regular
	Expression: pengetahuan mengenai regular expression adalah sangat
	penting di dunia open source, guna memanipulasi teks, pengenalan
	symbol, dan melakukan perintah-perintah secara batch.
	\item
	SAM/Applications: Client services. Bahasa pemrograman Python:
	mengenalkan interpreter Python sebagai bahasa script yang banyak
	manfaatnya untuk segala bidang seperti jaringan komputer, grafik,
	multithreading, database dan client/server.
	\item
	WTD/Web Technologies: Server-side programming, Web servers. Web
	development: membuat web server dengan Python.
	\item
	WTD/Web Technologies: Web services: mengenalkan akses ke web services
	seperti membaca harga saham dari Yahoo Finance.
	\item
	WTD/Web Development: Web interfaces, Website implementation and
	integration. Web framework dengan Python Django: membuat web dinamis
	dengan framework Django yang menggunakan Python.
	\item
	EDS/Enterprise Deployment Software: Configuration, definition and
	management. Cloud Computing: mengenalkan teknologi cloud computing
	yang berbasis open source, dengan memanfaatkan Google App Engine API.
\end{enumerate}

\vskip.15in
Pokok bahasan dari Matakuliah Teknologi Open Source akan disampaikan
dalam bentuk perkuliahan dan praktek di dalam laboratorium komputer. Pertemuan dalam matakuliah ini disusun sebagai berikut:

\begin{itemize}
	\item
	Bobot MK Teknologi Open Source = 3 sks.
	\item
	Total kuliah: 14 kali pertemuan/semester yang setara dengan 3 sks
\end{itemize}

\vskip.15in
\noindent\textbf{Rancangan Pembelajaran:}

\begin{longtable}{p{1cm}P{3cm}P{3cm}P{3cm}P{3.5cm}P{1.5cm}}
	\toprule
	\textbf{Perte-muan ke} & \textbf{Kemampuan akhir yang diharapkan } &
	\textbf{Materi Pembelajaran} & \textbf{Bentuk Pembelajaran} &
	\textbf{Kriteria Penilaian} & \textbf{Bobot Nilai (\%)}\\
	\midrule
	\endhead

1 & Memahami konsep-konsep dasar Open Source & 
\begin{myitemize}
	\item
		Arti open source
	\item
		Sejarah Linux
	\item
		Instalasi OS Linux Ubuntu dalam virtual machine
	\item
		Membuat akun Git-Hub atau BitBucket
\end{myitemize} & 
\begin{myitemize}
	\item
		Presentasi
	\item
		Tanya-jawab
	\item
		Demo/praktek instalasi OS
\end{myitemize} & &\\

2, 3 & Memahami perintah-perintah dasar OS Linux & 
\begin{myitemize}
	\item
	Perintah-perintah command line OS Linux
	\item
	Perintah-perintah dasar git
	\item
	Shell programming dengan bash
\end{myitemize} & 
\begin{myitemize}
	\item
	Presentasi
	\item
	Contoh kasus
	\item
	Tanya-jawab
	\item
	Latihan Tertulis di setiap akhir pertemuan
\end{myitemize} & 
\begin{myitemize} 
	\item
	Kemampuan memahami metode yang diajarkan dan dapat menerapkan untuk menjawab soal-soal latihan 
\end{myitemize} & 
Total Nilai Latihan = 25\% (*)\\

4, 5 & Memahami dan dapat menerapkan string manipulation, searching, menggunakan regular expression. & 
\begin{myitemize}
	\item
	Regular Expression dalam bash shell dan command line
	\item
	Regular Expression dalam Perl
	\item
	Pengenalan LaTeX.
\end{myitemize} & 
\begin{myitemize}
	\item
	Presentasi
	\item
	Contoh kasus
	\item
	Tanya-jawab
	\item
	Latihan Tertulis disetiap akhir pertemuan
\end{myitemize} & 
\begin{myitemize}
	\item Kemampuan memahami metode yang diajarkan dan dapat menerapkan untuk menjawab soal-soal latihan 
\end{myitemize} & 
Total Nilai Latihan = 25\% (*)\\

6, 7, 8 & Memahami dan dapat menerapkan pemrograman script Python &
\begin{myitemize}
	\item
	Pemrograman Python
	\item
	Pemrograman Python untuk jaringan
	\item
	Pemrograman Python untuk GUI
\end{myitemize} &
\begin{myitemize}
	\item
	Presentasi
	\item
	Contoh kasus
	\item
	Tanya-jawab
	\item
	Latihan Tertulis di setiap akhir pertemuan
\end{myitemize} &
\begin{myitemize}
	\item
	Kemampuan memahami metode yang diajarkan dan dapat menerapkan untuk menjawab soal-soal latihan
\end{myitemize} & 
Total Nilai Latihan = 25\% (*) \\

\textbf{9, 10 (UTS)} & & Materi UTS:
\begin{myitemize}
	\item
	Command line
	\item
	git
	\item
	Regular Expression
	\item
	Python scripting
\end{myitemize} &
\begin{myitemize}
	\item
	Open book, laptop/komputer
	\item
	180 menit
\end{myitemize} &
\begin{myitemize}
	\item Kemampuan memahami metode yang diajarkan 
\end{myitemize} &
25\%\\

11, 12 & Memahami dan dapat menerapkan Python sebagai web server &
\begin{myitemize}
	\item
	Python sebagai web server
\end{myitemize} & 
\begin{myitemize}
	\item
	Presentasi
	\item
	Contoh kasus
	\item
	Tanya-jawab
	\item
	Latihan Tertulis di setiap akhir pertemuan
\end{myitemize} &
\begin{myitemize}
	\item Kemampuan memahami metode yang diajarkan dan dapat menerapkan untuk menjawab soal-soal latihan
\end{myitemize} & 
Total Nilai Latihan = 25\% (*)\\

13 & Memahami dan dapat menerapkan proses Python web server yang melayani web service & \begin{myitemize}
	\item
	Web service
	\item
	XML
	\item
	REST
	\item
	JSON
\end{myitemize} & 
\begin{myitemize}
	\item
	Presentasi
	\item
	Contoh kasus
	\item
	Tanya-jawab
	\item
	Latihan Tertulis
\end{myitemize} &
\begin{myitemize}
	\item Kemampuan memahami metode yang diajarkan dan dapat menerapkan untuk menjawab soal-soal latihan
\end{myitemize} &
Total Nilai Latihan = 25\% (*)\\

14, 15, 16 & 
\begin{myitemize}
	\item 
	Memahami dan dapat memahami Python Django sebagai framework untuk membuat aplikasi berbasis web
	\item
	Mendiskusi-kan dalam kelompok untuk mengimplementasi kasus dalam sebuah proyek (UAS)
\end{myitemize} & 
\begin{myitemize}
	\item
	Python Django
\end{myitemize} & 
\begin{myitemize}
	\item
	Presentasi
	\item
	Contoh kasus
	\item
	Tanya-jawab
	\item
	Diskusi proyek akhir (UAS)
	\item
	Per kelompok maksimum 4 orang
\end{myitemize} & 
\begin{myitemize}
	\item
	Kemampuan memahami materi yang ditugaskan
	\item
	Kemampuan mepresentasikan materi yang ditugaskan
	\item
	Kemampuan menjawab pertanyaan-pertanyaan yang muncul selama diskusi.
\end{myitemize} &
25\%\\

17, 18 (UAS) & & Materi UAS:
\begin{myitemize}
	\item
	Presentasi aplikasi berbasis Open Source
\end{myitemize} & 
\begin{myitemize}
	\item
	Ujian Akhir Semester
	\item
	Presentasi
	\item
	10 menit per kelompok
\end{myitemize} & 
\begin{myitemize}
	\item
	Kemampuan memahami metode yang diajarkan selama satu semester dan dapat menerapkan untuk menjawab soal-soal UAS 
\end{myitemize} &
25\%\\

\bottomrule
\end{longtable}
\clearpage
\vspace*{.15in}
\noindent\textbf{Deskripsi Tugas dan Ujian:}

\begin{longtable}[]{p{4cm}P{12cm}}
	\toprule
	Tipe dan Bobot & Penjelasan dan Rubrik Nilai\\
	\midrule
	\endhead
	
	Latihan harian \break Bobot 25\% & 
	\begin{myitemize}
			\item
			Ujian tertulis (Sifat : Open book dan laptop/komputer)
			\item
			Nilai latihan harian akan ditotal menjadi sebuah nilai.
			\item
			Kriteria Penilaian: Nilai perorangan
		\end{myitemize}
		\begin{itemize}[label={$\bullet$},noitemsep,topsep=0pt,partopsep=0pt]
			\item
			Dikerjakan dengan langkah yang benar dan sistematis: 80\%
			\item
			Hasil yang dicapai benar: 20\%
		\end{itemize}
	\tabularnewline
		
	UTS \break Bobot 25\% &
	\begin{myitemize}
		\item
		Ujian tertulis (Sifat : Open book dan laptop/komputer)
		\item
		Kriteria Penilaian: Nilai perorangan
	\end{myitemize}	
	\begin{itemize}[label={$\bullet$},noitemsep,topsep=0pt,partopsep=0pt]
		\item
		Dikerjakan dengan langkah yang benar dan sistematis: 80\%
		\item
		Hasil yang dicapai benar: 20\%
	\end{itemize}\tabularnewline
	
	Diskusi dan Presentasi \break Bobot 25\% &
	\begin{myitemize}
		\item
		Bentuk: Forum Diskusi kelompok dan presentasi
		\item
		Kriteria Penilaian: Nilai kelompok
	\end{myitemize}	
	\begin{itemize}[label={$\bullet$},noitemsep,topsep=0pt,partopsep=0pt]
		\item
		Memahami materi yang ditugaskan: 60\%
		\item
		Dapat mepresentasikan materi dengan baik: 20\%
		\item
		Mampu menjawab pertanyaan - pertanyaan: 20\%
	\end{itemize}\tabularnewline
	
	
	Proyek Akhir (UAS) \break Bobot 25\% & 
	\begin{myitemize}
		\item
		Berbentuk software aplikasi yang dibuat untuk menyelesaikan problem
		sederhana di dunia nyata.
		\item
		Menggunakan software berbasis open source.
		\item
		Dikumpulkan dan dipresentasikan pada dosen pada saat UAS.
		\item
		Kriteria Penilaian: Nilai kelompok
	\end{myitemize}
	
	\begin{itemize}[label={$\bullet$},noitemsep,topsep=0pt,partopsep=0pt]
		\item
		Software aplikasi berhasil dibuat: 25\%
		\item
		Sesuai dengan metode yang ditugaskan: 50\%
		\item
		Hasil yang dicapai benar: 25\%
	\end{itemize}\tabularnewline
	\bottomrule
\end{longtable}

\iffalse
\vskip.15in
\noindent\textbf{Tanggal-tanggal Penting:}
\begin{center} \begin{minipage}{3.8in}
\begin{flushleft}
Midterm \#1      \dotfill ~\={A}b\={a}n 16, 1393  \\
Midterm \#2      \dotfill ~\={A}zar 21, 1393  \\
%Project Deadline \dotfill ~Month Day \\
Final Exam       \dotfill ~Dey 18, 1393  \\
\end{flushleft}
\end{minipage}
\end{center}
\fi
\clearpage
\vskip.15in
\noindent\textbf{Norma Akademik:}
\begin{enumerate}
	\item
	Kegiatan pembelajaran dimulai tepat waktu.
	\item
	Selama proses pembelajaran berlangsung HP dimatikan/silent. Mahasiswa tidak diijinkan menggunakan/bermain HP di kelas.
	\item
	Pengumpulan tugas ditetapkan sesuai jadwal. Bagi yang terlambat nilai hanya 75\% nya, dan bila terlambat 1 hari mendapat nilai hanya 50\% nya, lebih dari satu hari mendapat nilai 0\%.
	\item
	Mahasiswa yang mengerjakan tugas dan ujian yang terbukti mencontoh mahasiswa lain diminta mengundurkan diri semester ini dan bisa mengikuti di semester depan (bila ada).
	\item 
	Aturan jumlah minimal presensi dalam pembelajaran tetap diberlakukan
	(75\% kehadiran), termasuk aturan cara berpakaian atau bersepatu.

\end{enumerate}
%%%%%% THE END
\end{document}
